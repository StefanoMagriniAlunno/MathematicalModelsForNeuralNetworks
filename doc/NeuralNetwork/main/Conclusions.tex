\chapter{Conclusion}

% recap generale
Although discrete Hopfield networks are a fundamental model in the field of neural networks, they have structural limitations that limit their applicability in modern deep learning architectures. However, their ability to store and retrieve even partially damaged patterns makes them useful in contexts such as pattern recognition and error correction, especially in resource-constrained environments. Acting as "associative memories", they store and retrieve patterns in the presence of partial or noisy inputs, making them particularly effective in areas such as image recognition or error correction in telecommunications. Their rapid stabilization toward an equilibrium state also makes them useful in models of biological behavior. However, their limited memory capacity and inability to be trained by backpropagation reduce their effectiveness in more complex applications.

\noindent Despite their advantages, discrete Hopfield networks suffer from significant limitations in their storage capacity and lack of trainability via backpropagation methods, making them unsuitable for more complex applications. Pattern storage is limited by the fact that the maximum number of patterns that can be stored increases linearly with the number of neurons, making the model ineffective on a large scale. In addition, the lack of gradient descent trainability prevents optimization for more sophisticated applications. These limitations have led to the development of continuous Hopfield networks, which offer greater flexibility, better generalization, and a wide range of applications, especially when integrated with other neural layers.

% Applicazioni e risultati
\noindent To overcome the limitations of discrete networks, continuous Hopfield networks have proven to be a powerful tool, especially when integrated with layers that exploit their properties. In our analysis, we have experimented with several structures that apply the principles of continuous networks to achieve remarkable results. The \texttt{DeepHNN} layer allows multiple patterns to be associated with a single feature, improving the robustness of information retrieval, while \texttt{ConvHNN} overcomes the problem of pattern translation through local analysis, which is more effective than traditional convolutions at removing noise. The \texttt{VHNN model}, which combines continuous Hopfield networks and variational autoencoders (VAE), allows more efficient training and provides theoretical insights into the search for energy minima in stochastic processes. Although continuous Hopfield networks are simpler than other neural architectures, their small number of parameters makes them ideal for resource-constrained environments. In addition, their clustering capability makes them more resistant to overfitting, which improves generalization. However, the training speed is slower than for more complex models and requires optimizers with high learning rates and moderate batch sizes for good convergence, with computational cost increasing with the number of iterations.

% Conclusioni e sviluppi futuri
\noindent A possible future development concerns the theoretical analysis of the behavior of variational Hopfield networks (VHNNs), which raises numerous theoretical questions, such as Is there a limit distribution for stochastic processes in Hopfield networks? What is the waiting time before the network changes state, and how is it related to the energy associated with a state? How do Hopfield networks behave as the number of neurons (or pixels) tends to infinity? Answering these questions could lead to significant theoretical developments in the field of complex dynamical systems and stochastic neural networks, improve the training efficiency of Hopfield networks, and open new avenues for applications in systems theory, optimization, and theoretical machine learning.
